\documentclass[]{article}
\usepackage{lmodern}
\usepackage{amssymb,amsmath}
\usepackage{ifxetex,ifluatex}
\usepackage{fixltx2e} % provides \textsubscript
\ifnum 0\ifxetex 1\fi\ifluatex 1\fi=0 % if pdftex
  \usepackage[T1]{fontenc}
  \usepackage[utf8]{inputenc}
\else % if luatex or xelatex
  \ifxetex
    \usepackage{mathspec}
  \else
    \usepackage{fontspec}
  \fi
  \defaultfontfeatures{Ligatures=TeX,Scale=MatchLowercase}
\fi
% use upquote if available, for straight quotes in verbatim environments
\IfFileExists{upquote.sty}{\usepackage{upquote}}{}
% use microtype if available
\IfFileExists{microtype.sty}{%
\usepackage{microtype}
\UseMicrotypeSet[protrusion]{basicmath} % disable protrusion for tt fonts
}{}
\usepackage[margin=1in]{geometry}
\usepackage{hyperref}
\hypersetup{unicode=true,
            pdftitle={NZ GREEN Grid project example:},
            pdfauthor={Ben Anderson (b.anderson@soton.ac.uk, @dataknut)},
            pdfborder={0 0 0},
            breaklinks=true}
\urlstyle{same}  % don't use monospace font for urls
\usepackage{longtable,booktabs}
\usepackage{graphicx,grffile}
\makeatletter
\def\maxwidth{\ifdim\Gin@nat@width>\linewidth\linewidth\else\Gin@nat@width\fi}
\def\maxheight{\ifdim\Gin@nat@height>\textheight\textheight\else\Gin@nat@height\fi}
\makeatother
% Scale images if necessary, so that they will not overflow the page
% margins by default, and it is still possible to overwrite the defaults
% using explicit options in \includegraphics[width, height, ...]{}
\setkeys{Gin}{width=\maxwidth,height=\maxheight,keepaspectratio}
\IfFileExists{parskip.sty}{%
\usepackage{parskip}
}{% else
\setlength{\parindent}{0pt}
\setlength{\parskip}{6pt plus 2pt minus 1pt}
}
\setlength{\emergencystretch}{3em}  % prevent overfull lines
\providecommand{\tightlist}{%
  \setlength{\itemsep}{0pt}\setlength{\parskip}{0pt}}
\setcounter{secnumdepth}{5}
% Redefines (sub)paragraphs to behave more like sections
\ifx\paragraph\undefined\else
\let\oldparagraph\paragraph
\renewcommand{\paragraph}[1]{\oldparagraph{#1}\mbox{}}
\fi
\ifx\subparagraph\undefined\else
\let\oldsubparagraph\subparagraph
\renewcommand{\subparagraph}[1]{\oldsubparagraph{#1}\mbox{}}
\fi

%%% Use protect on footnotes to avoid problems with footnotes in titles
\let\rmarkdownfootnote\footnote%
\def\footnote{\protect\rmarkdownfootnote}

%%% Change title format to be more compact
\usepackage{titling}

% Create subtitle command for use in maketitle
\newcommand{\subtitle}[1]{
  \posttitle{
    \begin{center}\large#1\end{center}
    }
}

\setlength{\droptitle}{-2em}

  \title{NZ GREEN Grid project example:}
    \pretitle{\vspace{\droptitle}\centering\huge}
  \posttitle{\par}
  \subtitle{Testing power demand: rf\_38}
  \author{Ben Anderson
(\href{mailto:b.anderson@soton.ac.uk}{\nolinkurl{b.anderson@soton.ac.uk}},
\texttt{@dataknut})}
    \preauthor{\centering\large\emph}
  \postauthor{\par}
      \predate{\centering\large\emph}
  \postdate{\par}
    \date{Last run at: 2018-08-20 22:54:48}


\usepackage{amsthm}
\newtheorem{theorem}{Theorem}[section]
\newtheorem{lemma}{Lemma}[section]
\theoremstyle{definition}
\newtheorem{definition}{Definition}[section]
\newtheorem{corollary}{Corollary}[section]
\newtheorem{proposition}{Proposition}[section]
\theoremstyle{definition}
\newtheorem{example}{Example}[section]
\theoremstyle{definition}
\newtheorem{exercise}{Exercise}[section]
\theoremstyle{remark}
\newtheorem*{remark}{Remark}
\newtheorem*{solution}{Solution}
\begin{document}
\maketitle

{
\setcounter{tocdepth}{2}
\tableofcontents
}
\newpage

\section{About}\label{about}

\subsection{Report circulation:}\label{report-circulation}

\begin{itemize}
\tightlist
\item
  Public - this report is intended to accompany the data release.
\end{itemize}

\subsection{License}\label{license}

This work is made available under the Creative Commons
\href{https://creativecommons.org/licenses/by-sa/4.0/}{Attribution-ShareAlike
4.0 International (CC BY-SA 4.0) License}.

This means you are free to:

\begin{itemize}
\tightlist
\item
  \emph{Share} --- copy and redistribute the material in any medium or
  format
\item
  \emph{Adapt} --- remix, transform, and build upon the material for any
  purpose, even commercially.
\end{itemize}

Under the following terms:

\begin{itemize}
\tightlist
\item
  \emph{Attribution} --- You must give appropriate credit, provide a
  link to the license, and indicate if changes were made. You may do so
  in any reasonable manner, but not in any way that suggests the
  licensor endorses you or your use.
\item
  \emph{ShareAlike} --- If you remix, transform, or build upon the
  material, you must distribute your contributions under the same
  license as the original.
\item
  \emph{No additional restrictions} --- You may not apply legal terms or
  technological measures that legally restrict others from doing
  anything the license permits.
\end{itemize}

\textbf{Notices:}

\begin{itemize}
\tightlist
\item
  You do not have to comply with the license for elements of the
  material in the public domain or where your use is permitted by an
  applicable exception or limitation.
\item
  No warranties are given. The license may not give you all of the
  permissions necessary for your intended use. For example, other rights
  such as publicity, privacy, or moral rights may limit how you use the
  material. \#YMMV
\end{itemize}

For the avoidance of doubt and explanation of terms please refer to the
full \href{https://creativecommons.org/licenses/by-sa/4.0/}{license
notice} and
\href{https://creativecommons.org/licenses/by-sa/4.0/legalcode}{legal
code}.

\subsection{Citation}\label{citation}

If you wish to use any of the material from this report please cite as:

\begin{itemize}
\tightlist
\item
  Anderson, B. (2018) NZ GREEN Grid project example: Testing power
  demand: rf\_38
  \href{http://www.otago.ac.nz/centre-sustainability/}{Centre for
  Sustainability}, University of Otago: Dunedin.
\end{itemize}

This work is (c) 2018 the University of Southampton.

\subsection{History}\label{history}

\begin{itemize}
\tightlist
\item
  \href{https://github.com/CfSOtago/GREENGridData/commits/master/examples/testHouseholdPower.Rmd}{Report
  history}
\end{itemize}

\subsection{Requirements:}\label{requirements}

This report uses the safe version of the grid spy 1 minute data which
has been processed using
\url{https://github.com/CfSOtago/GREENGridData/tree/master/dataProcessing/gridSpy}.

\subsection{Support}\label{support}

This work was supported by:

\begin{itemize}
\tightlist
\item
  The \href{https://www.otago.ac.nz/}{University of Otago};
\item
  The \href{https://www.southampton.ac.uk/}{University of Southampton};
\item
  The New Zealand \href{http://www.mbie.govt.nz/}{Ministry of Business,
  Innovation and Employment (MBIE)} through the
  \href{https://www.otago.ac.nz/centre-sustainability/research/energy/otago050285.html}{NZ
  GREEN Grid} grant (Contract ID: UOCX1203);
\item
  \href{http://www.energy.soton.ac.uk/tag/spatialec/}{SPATIALEC} - a
  \href{http://ec.europa.eu/research/mariecurieactions/about-msca/actions/if/index_en.htm}{Marie
  Skłodowska-Curie Global Fellowship} based at the University of Otago's
  \href{http://www.otago.ac.nz/centre-sustainability/staff/otago673896.html}{Centre
  for Sustainability} (2017-2019) \& the University of Southampton's
  Sustainable Energy Research Group (2019-202).
\end{itemize}

\newpage

\section{Introduction}\label{introduction}

The \href{https://cfsotago.github.io/GREENGridData/}{NZ GREEN Grid
household electricity demand study} recruited a sample of c 25
households in each of two regions of New Zealand ({\textbf{???}}). The
first sample was recruited in early 2014 and the second in early 2015.
Research data includes:

\begin{itemize}
\tightlist
\item
  1 minute electricity power (W) data was collected for each dwelling
  circuit using \href{https://gridspy.com/}{GridSpy} monitors on each
  power circuit (and the incoming power). The power values represent
  mean(W) over the minute preceeding the observation timestamp;
\item
  Dwelling \& appliance surveys;
\item
  Occupant time-use diaries (focused on energy use).
\end{itemize}

NB: Version 1 of the data package does not include the time-use diaries.

This report provides summary analysis of one household as an example.

\section{Load rf\_38 data}\label{load-rf_38-data}

The data used to generate this report is:

\begin{itemize}
\tightlist
\item
  /Volumes/hum-csafe/Research Projects/GREEN
  Grid/cleanData/safe/gridSpy/1min/data/rf\_38\_all\_1min\_data.csv.gz
\item
  /Volumes/hum-csafe/Research Projects/GREEN
  Grid/cleanData/safe/survey/ggHouseholdAttributesSafe.csv
\end{itemize}

\begin{verbatim}
## Parsed with column specification:
## cols(
##   .default = col_integer(),
##   linkID = col_character(),
##   hasApplianceSummary = col_character(),
##   Oven = col_character(),
##   `Fridge / Freezer 1` = col_character(),
##   `Fridge / Freezer 2` = col_character(),
##   `Fridge / Freezer 3` = col_character(),
##   Dishwasher = col_character(),
##   Microwave = col_character(),
##   `Washing Machine` = col_character(),
##   `Clothes Dryer` = col_character(),
##   `Hot water cylinder` = col_character(),
##   `Other Appliance` = col_character(),
##   `Electric heater` = col_character(),
##   `Heated towel rails` = col_character(),
##   `PV Inverter` = col_character(),
##   `Energy Storage` = col_character(),
##   `Other Generation Device` = col_character(),
##   hasLongSurvey = col_character(),
##   Q14_1 = col_double(),
##   Q19_2 = col_character()
##   # ... with 11 more columns
## )
\end{verbatim}

\begin{verbatim}
## See spec(...) for full column specifications.
\end{verbatim}

\begin{table}

\caption{\label{tab:hhData}Summary of household data forrf_38}
\centering
\begin{tabular}[t]{lrrllll}
\toprule
linkID & nAdults & nTeenagers13\_18 & Location & hasLongSurvey & hasShortSurvey & hasApplianceSummary\\
\midrule
rf\_38 & 2 & 0 & Hawkes Bay & NA & Yes & Yes\\
\bottomrule
\end{tabular}
\end{table}

Table \ref{tab:hhData} shows household attributes such as how many
people live in this household.

\begin{table}

\caption{\label{tab:gsData}Summary of grid spy data forrf_38}
\centering
\begin{tabular}[t]{l|l|l|l|l|l|l|l}
\hline
  &     hhID &    linkID & dateTime\_orig &   TZ\_orig &   r\_dateTime &   circuit &     powerW\\
\hline
 & Length:5319113 & Length:5319113 & Length:5319113 & Length:5319113 & Min.   :2015-03-25 03:51:00 & Length:5319113 & Min.   :-179.0\\
\hline
 & Class :character & Class :character & Class :character & Class :character & 1st Qu.:2015-08-28 14:06:00 & Class :character & 1st Qu.:   0.0\\
\hline
 & Mode  :character & Mode  :character & Mode  :character & Mode  :character & Median :2016-10-18 01:51:00 & Mode  :character & Median :  50.8\\
\hline
 & NA & NA & NA & NA & Mean   :2016-06-21 10:31:25 & NA & Mean   : 263.7\\
\hline
 & NA & NA & NA & NA & 3rd Qu.:2017-03-21 06:54:00 & NA & 3rd Qu.: 170.2\\
\hline
 & NA & NA & NA & NA & Max.   :2017-08-22 06:37:00 & NA & Max.   :6678.4\\
\hline
\end{tabular}
\end{table}

Table \ref{tab:gsData} shows a summary of the grid spy 1 minute power
data.

Note that:

\begin{itemize}
\tightlist
\item
  the original dataTime (\texttt{dateTime\_orig}) and TZ
  (\texttt{TZ\_orig}) have been retained so that the user can check for
  parsing errors (see
  \url{https://github.com/dataknut/nzGREENGridDataR/issues/2}) if
  required;
\item
  r\_datetime is the correct dateTime of each observation in UTC and
  will have loaded as your local timezone. If you are conducting this
  analysis outside NZ then you will get strange results until you use
  \href{https://lubridate.tidyverse.org/}{lubridate} to tell R to use tz
  = ``Pacific/Auckland'' with this variable;
\end{itemize}

\section{Plot monthly power profiles}\label{plot-monthly-power-profiles}

This section plots overall mean power (W) per minute per month for each
circuit to show:

\begin{itemize}
\tightlist
\item
  patterns of missing data (no lines)
\item
  patterns of consumption
\item
  possible dateTime issues (where consumption patterns seem to be
  stangely shifted in time)
\item
  possible PV installation
\end{itemize}

\begin{figure}
\centering
\includegraphics{testHouseholdPower_files/figure-latex/plotProfiles-1.pdf}
\caption{\label{fig:plotProfiles}Demand profile plot}
\end{figure}

Figure \ref{fig:plotProfiles} shows the plot for this household
(rf\_38). Can you see anything interesting or unusual?

\section{Runtime}\label{runtime}

Analysis completed in 49.75 seconds ( 0.83 minutes) using
\href{https://cran.r-project.org/package=knitr}{knitr} in
\href{http://www.rstudio.com}{RStudio} with R version 3.5.1 (2018-07-02)
running on x86\_64-apple-darwin15.6.0.

\section{R environment}\label{r-environment}

R packages used:

\begin{itemize}
\tightlist
\item
  base R - for the basics (R Core Team 2016)
\item
  data.table - for fast (big) data handling (Dowle et al. 2015)
\item
  lubridate - date manipulation (Grolemund and Wickham 2011)
\item
  ggplot2 - for slick graphics (Wickham 2009)
\item
  readr - for csv reading/writing (Wickham, Hester, and Francois 2016)
\item
  knitr - to create this document \& neat tables (Xie 2016)
\item
  nzGREENGridDataR - for local NZ GREEN Grid project utilities
\end{itemize}

Session info:

\begin{verbatim}
## R version 3.5.1 (2018-07-02)
## Platform: x86_64-apple-darwin15.6.0 (64-bit)
## Running under: macOS High Sierra 10.13.6
## 
## Matrix products: default
## BLAS: /Library/Frameworks/R.framework/Versions/3.5/Resources/lib/libRblas.0.dylib
## LAPACK: /Library/Frameworks/R.framework/Versions/3.5/Resources/lib/libRlapack.dylib
## 
## locale:
## [1] en_NZ.UTF-8/en_NZ.UTF-8/en_NZ.UTF-8/C/en_NZ.UTF-8/en_NZ.UTF-8
## 
## attached base packages:
## [1] stats     graphics  grDevices utils     datasets  methods   base     
## 
## other attached packages:
## [1] knitr_1.20        readr_1.1.1       ggplot2_3.0.0     lubridate_1.7.4  
## [5] data.table_1.11.4 GREENGridData_1.0
## 
## loaded via a namespace (and not attached):
##  [1] Rcpp_0.12.18      highr_0.7         pillar_1.3.0     
##  [4] compiler_3.5.1    cellranger_1.1.0  plyr_1.8.4       
##  [7] bindr_0.1.1       prettyunits_1.0.2 tools_3.5.1      
## [10] progress_1.2.0    digest_0.6.15     gtable_0.2.0     
## [13] evaluate_0.11     tibble_1.4.2      pkgconfig_2.0.1  
## [16] rlang_0.2.1       rstudioapi_0.7    yaml_2.2.0       
## [19] xfun_0.3          bindrcpp_0.2.2    withr_2.1.2      
## [22] dplyr_0.7.6       stringr_1.3.1     hms_0.4.2        
## [25] grid_3.5.1        rprojroot_1.3-2   tidyselect_0.2.4 
## [28] glue_1.3.0        R6_2.2.2          readxl_1.1.0     
## [31] rmarkdown_1.10    bookdown_0.7      purrr_0.2.5      
## [34] reshape2_1.4.3    magrittr_1.5      scales_0.5.0     
## [37] backports_1.1.2   htmltools_0.3.6   assertthat_0.2.0 
## [40] colorspace_1.3-2  labeling_0.3      stringi_1.2.4    
## [43] lazyeval_0.2.1    munsell_0.5.0     crayon_1.3.4
\end{verbatim}

\section*{References}\label{references}
\addcontentsline{toc}{section}{References}

\hypertarget{refs}{}
\hypertarget{ref-data.table}{}
Dowle, M, A Srinivasan, T Short, S Lianoglou with contributions from R
Saporta, and E Antonyan. 2015. \emph{Data.table: Extension of
Data.frame}. \url{https://CRAN.R-project.org/package=data.table}.

\hypertarget{ref-lubridate}{}
Grolemund, Garrett, and Hadley Wickham. 2011. ``Dates and Times Made
Easy with lubridate.'' \emph{Journal of Statistical Software} 40 (3):
1--25. \url{http://www.jstatsoft.org/v40/i03/}.

\hypertarget{ref-baseR}{}
R Core Team. 2016. \emph{R: A Language and Environment for Statistical
Computing}. Vienna, Austria: R Foundation for Statistical Computing.
\url{https://www.R-project.org/}.

\hypertarget{ref-ggplot2}{}
Wickham, Hadley. 2009. \emph{Ggplot2: Elegant Graphics for Data
Analysis}. Springer-Verlag New York. \url{http://ggplot2.org}.

\hypertarget{ref-readr}{}
Wickham, Hadley, Jim Hester, and Romain Francois. 2016. \emph{Readr:
Read Tabular Data}. \url{https://CRAN.R-project.org/package=readr}.

\hypertarget{ref-knitr}{}
Xie, Yihui. 2016. \emph{Knitr: A General-Purpose Package for Dynamic
Report Generation in R}. \url{https://CRAN.R-project.org/package=knitr}.


\end{document}
